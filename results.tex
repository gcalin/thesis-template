\chapter{\label{cha:results}Results}

The aim of this thesis is to gain insight into the behavior and applicability
of heuristics to task of differential testing of the Kotlin compiler.
To achieve this goal, this chapter seeks to answer the research questions posed in
\Cref{sec:rqs} through empirical evidence and statistical analysis.
We structure the analysis of the results as to mirror the hierarchy
of research questions.
\Cref{sec:resrq1} analyzes the effect of the the most impactful
hyperparameters and the ramifications on the generative heuristics.
\Cref{sec:resrq2} details the performance of the best representative
of each heuristic category.

\section{\label{sec:resrq1}Influence of Guidance Parameters}

We begin our analysis of hyperparameter influence by considering the effects of
the simplicity bias on the structure, size, and number of files that \kf~ generates
in \Cref{subsec:simpl_bias}, followed by the effects of distance metrics on diversity
heuristics in \Cref{subsec:distance_effect}, and finally the impact 
of target selection in \Cref{subsec:targets_effect}.

\subsection{\label{subsec:simpl_bias}Effects of Simplicity Bias on Random Sampling}

To study the effects of the simplicity bias, we focus on the size and number of files
generated by \gls{RS}.

\subsection{\label{subsec:distance_effect}Effects of Distance Metric on Syntactic Diversity-Driven Search}

\subsection{\label{subsec:targets_effect}Effects of Target Set on Semantic Proximity-Driven Search}


\section{\label{sec:resrq2}Performance of Heuristics}