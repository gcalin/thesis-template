Kotlin is a programming language best known for
its interoperability with Java, as well as the measurable improvements it offers over it.
Since its designation as Android's go-to language in 2019, the popularity
and impact of Kotlin have risen greatly.
% Hundreds of thousands of developers actively use Kotlin,
% which is the primary language of over 1 million open-source repositories.
Amidst this surge in popularity, the Kotlin
developer team is working on a new version of the compiler that introduces
sweeping changes to the ecosystem.

Traditional compiler testing is a manual and laborious task
that requires extensive developer effort and expertise.
In an attempt to mitigate this, researchers have invested
great resources in developing and perfecting automated compiler testing tools over the last decades.
These approaches generate new pieces of code to test the behavior
of compilers, which is assessed through differential testing.
However, the usage of heuristics as guidance for the generative process
is not well understood, and no approach that generates Kotlin code from scratch currently exists.

In this thesis, we propose a novel method
of enriching standard grammar specifications with language-targeted semantic
context that is integrated in the sampling process.
We structure generated code hierarchically
and use it as the base of an evolutionary computation framework.
% We structure the code that emerges from sampling
% the enriched grammar structure into a hierarchical
% representation based on scope and complexity.
% We then use this representation as the basis
% of an evolutionary framework that provides guidance
% to the sampling process.
Within this framework, we introduce two
classes of algorithms that are novel to the field
of compiler fuzzing, based on syntactic diversity and semantic proximity, respectively.
% We introduce both single- and many-objective
% formulations of both heuristics, in addition to
% the literature-standard random sampling.

We carry out an  empirical analysis spanning 200K generated Kotlin files, 
which we analyzed through different Kotlin compiler versions.
Our results uncovered five previously unreported categories of bugs,
which we reported to the Kotlin compiler developer team.
The developers verified and replicated our instances
on the current release of the Kotlin compiler,
and have assigned target release dates for fixes within the current
major version of the compiler.
The study also provides new insight into
the effects of heuristic-specific hyperparameters such as
expression simplicity, dissimilarity measurements, and target selection.
% The study also shows the driving mechanisms of our algorithms
% favor different code patterns, which in turn give rise
% to different defects in the new Kotlin compiler.


