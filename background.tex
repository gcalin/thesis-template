\chapter{\label{cha:background}Background} 

This chapter establishes the theoretical basis
required for investigating the aim introduced in
Section \ref{sec:aim}.
This includes the individual nuances of the Kotlin compiler,
the encompassing field of software testing, and the theoretical
background of fuzzing.
In addition, this chapter lays a contextual and historical foundation
required for the analysis of related work examined in Chapter \ref{cha:related}.

\section{The Kotlin Compiler}

%\newglossaryentry{formula}
%{
%        name=formula,
%        description={A mathematical expression}
%}
%\Gls{formula}
The Kotlin project began in 2010, with version \texttt{1.0}
officially releasing six years later \citep{stepanov2021type}.
Following Google's decision to support it as an official Android
programming language \cite{kotlin5years}, Kotlin's popularity increased drastically.
This exceptional growth is perhaps best exemplified through
StackOverflow's annual developer survey, which aggregates data from
tens of thousands of active developers from varied backgrounds.
This survey clearly showcases Kotlin's sharp ascent: the percentage of
developers who actively use Kotlin increased from $0.12$ per cent in 2016
to $9.16$ per cent in 2022 \citep{sosurvey2016, sosurvey2022}. 
Such sustained growth brought both opportunities and challenges to the
expanding community, which resulted in the development of increasingly
complex and feature-rich tools.

Such changes have not gone unnoticed by Kotlin developer
community, who for the past seven years has introduced numerous
changes to all parts of the language ecosystem to
accommodate changes in the scope of the language.
Inevitably, the compiler too must adapt to the direction that the ecosystem is taking.
Perhaps one of the biggest changes to Kotlin as a whole
is the upcoming release of version 2.0 of the language,
complete with a sweeping re-write of the compiler
frontend \footnote{https://blog.jetbrains.com/kotlin/2023/02/k2-kotlin-2-0/}.
This component is responsible for parsing, analyzing, and translating source code
into an intermediate representation.

Code-named \texttt{K2}, the new version of the compiler seeks
to increase performance and maintain backwards compatibility,
all while entirely replacing the architecture of the current compiler frontend.
Naturally, any defects in the new compiler implementation
could have drastic repercussions for Kotlin users, which coupled with the
sheer size of the rewrite, warrants extensive testing.
The remainder of this study targets the testing of the new
(experimental) implementation of the \texttt{K2} compiler
against its previous counterpart, with the goal
of uncovering defects that the introduction of \texttt{K2} brings with it.

\newacronym{JVM}{JVM} {Java Virtual Machine}

Though it supports several platforms, large portions of the Kotlin ecosystem pivot around
the \Gls{JVM} system and its adjacent Java bytecode instruction set.
Kotlin is one of several successful languages to leverage the \gls{JVM},
together with Java, Scala, Clojure, and Groovy. 
One can think of the \gls{JVM} as a compatibility layer
between a Java bytecode input and a target language.
The advantage of this system lies in the flexibility it affords
language engineers.
Leveraging the \gls{JVM} allows
language designers to focus on building a single compilation tool
that targets Java bytecode.
This approach delegates the task of machine code
generation to the \gls{JVM} and frees compiler engineers from
platform- and hardware-specific idiosyncrasies.
This study focuses on testing the end-to-end compilation pipeline
on the \gls{JVM} platform
of Kotlin through the open-source \texttt{kotlinc} utility.

\newacronym{SDLC}{SDLC} {Software Development Lifecycle}
\newacronym{SUT}{SUT} {System Under Test}
\newacronym{WB}{WB} {White-Box}
\newacronym{BB}{BB} {Black-Box}
\newacronym{GB}{GB} {Grey-Box}
\newacronym{UT}{UT} {Unit-Level Testing}
\newacronym{IT}{IT} {Integration-Level Testing}
\newacronym{ST}{ST} {System-Level Test}
\newacronym{DT}{DT} {Differential Testing}

\section{Software Testing}

Testing is an integral component of the \gls{SDLC} \cite{jamil2016software}.
At its core, software testing is the process of validating
the behavior of a target \gls{SUT}. 
Over the years, researchers expanded vast resources into
developing, and understanding the nuances of
software testing, both manual and automated.
To effectively navigate the numerous alternatives, we use the
taxonomy established by \citet{umar2020comprehensive}.

\subsubsection{Black-, White-, and Grey-Box Testing}

A key characteristic of a testing strategy consists of 
the requirements it imposes on the \gls{SUT}. 
\Gls{WB} approaches require a full understanding of the target
system, which often translates to unrestricted access to its source code.
The information extracted through access to the system's structure
and implementation often provides valuable guidance criteria.
By contrast, \Gls{BB} approaches function without access to the underlying
source code, which diminishes the amount of information they can exploit.
This sacrifice, however, extends the applicability of \gls{BB} approaches 
to a broader range of systems.
\Gls{GB} approaches constitute a trade-off between the two 
extremes, and assume access to a subset system components.
This requires some level of understanding of the \gls{SUT}'s
behavior, but imposes lesser constraints than \gls{WB} standards.

\subsubsection{Unit-, Integration-, and System-Level Testing}

Testing methods also differ in the degree to which they exercise the \gls{SUT}.
Three main paradigms emerge from this distinction: 
\Gls{UT}, \Gls{IT}, and \gls{ST}.
\citet{daka2014survey} demarcate \gls{UT} as 
generally small and automatically executable tests that aim
to validate the correctness of small components (units) of the \gls{SUT}.
\citet{leung1990study} establish \gls{IT} at a higher-level scope, targeting
the integration of several smaller modules within a system.
Finally, \citet{umar2020comprehensive} identifies \gls{ST} as tests
that validate the entire system's compliance against its underlying
requirements.

\subsubsection{The Oracle Problem}

Irrespective of access to the \gls{SUT}
and its level of granularity, test cases must be capable
of distinguishing between correct and faulty behavior.
This challenge constitutes the
\textit{test oracle problem}, and its automation is among the 
greatest obstacles in automating test case generation \citep{barr2014oracle}.
Approaches like specification inference and regression testing offer
approximate oracles that can partially automate this process.
Originally proposed by \citet{mckeeman1998differential} in 1998, \Gls{DT}
offers a compelling and versatile alternative, that uses different versions
of the same \gls{SUT} to heuristically discover bugs.
\gls{DT} identifies incorrect behavior when at least
one version of the \gls{SUT} returns a result that differs from the others.
While inexact in its nature, \gls{DT} has become popular
in the field of compiler testing thanks to its lightweight 
constraints, which increase its applicability to especially complex
pieces of software, like compiler ecosystems.

\section{Automated Test Case Generation}

Traditional (manual) software testing poses several practical
challenges, not least because of its imposing complexity.
\citet{candea2019automated} regard testing as the most
demanding and laborious task in the \gls{SDLC}.
\citet{lehman1979understanding} established that to maintain
relevance, software systems must adapt, while \citet{zaidman2011studying}
showed that test suites often evolve in tandem with their code bases.
These findings suggest that to adequately test an evolving system,
practitioners not only allocate significant resources to writing test cases,
but also do so throughout the software's lifetime.
In an effort to alleviate the human resources that traditional compiler testing requires,
researchers developed a plethora of automated test case generation methods.
The remainder of this section briefly covers the most prevalent techniques
in this field.

\newacronym{OS}{OS} {Operating System}
\newacronym{RT}{RT} {Random Testing}
\newacronym{ART}{ART} {Adaptive Random Testing}

\subsection{Fuzzing}

\citet{miller1990empirical} coined the term \textit{fuzz} in 1979 referring
to a program capable of generating a stream of random characters.
This program served as a means of assessing
the reliability of several tools that were part of the Unix \Gls{OS}.
At its core, fuzzing is perhaps the simplest possible implementation
of input generation: the \textit{fuzzer} simply samples data
driven by a pseudo-random random generator and executes
the target program with the generated input.
This original formulation of fuzzing bears a close
resemblance to well established test case generation algorithms
such as \Gls{RT} \cite{duran1981report} and 
its descendant, \Gls{ART} \cite{chen2010adaptive}.
Nonetheless, fuzzing is a straight-forward and deceptively effective 
tool for defect discovery, which led it to become one
of the most prevalent techniques in the field of test case generation.

Since its inception, the meaning of fuzzing has adapted to
better accommodate the extensive domain of applications that
benefited from its application.
Such applications include security testing \cite{oehlert2005violating, stephens2016driller},
data structure signature generation \cite{dolan2009robust}, 
database system management testing \cite{zhong2020squirrel},
and REST API testing \cite{atlidakis2019restler, godefroid2020intelligent}.
To encapsulate the various complexities that fuzzing has grown
to exhibit, \citet{manes2019art} define fuzzing as the execution
of the \gls{SUT} using input sampled from an input space protruding
that of the \gls{SUT}'s.
\citet{saavedra2019review} adopt a similar definition and additionally
establish a taxonomy, that distinguishes between three 
fundamental fuzzing paradigms:

\begin{itemize}
\item \textbf{Mutation-based fuzzers}~rely on predefined input (seeds),
which they manipulate in heuristic ways to better cover that \gls{SUT}'s input space.
The performance of these approaches is often heavily correlated
with the quality of the corpus of seed inputs as the main source
of exploiting program knowledge.


\item \textbf{Generation-based fuzzers}~sample the input space
guided by an a-priori configuration or specification of the \gls{SUT}.
Such approaches are generally more sophisticated than their mutation-based
counterparts, which they tend to outperform \cite{saavedra2019review}.
However, these benefits come at the cost of complexity and
adherence to the limitations provided through the configuration.

\item \textbf{Evolutionary fuzzers}~ employ evolutionary computation
techniques, including fitness evaluation, mutation, crossover, and selection,
to iteratively improve the quality of a set of solutions.
These algorithms benefit from the rich literature surrounding
evolutionary computation and may be better suited at exploring
specific volumes of the search space.
Evolutionary fuzzers impose an additional requirement on the \gls{SUT},
namely that it must be capable of meaningfully evaluating generated input.
\end{itemize}

\newacronym{SBSE}{SBSE} {Search-Based Software Engineering}
\newacronym{SBST}{SBST} {Search-Based Software Testing}

\subsection{Search-Based Software Testing}

Software engineering entails numerous problems that present
nuanced, multivariate, and often conflicting objectives.
Metaheuristic search algorithms lend themselves naturally
as frameworks for effectively exploring the space of possible
solutions with the aim of finding (near-)optimal candidates \cite{harman2001search}.
The study and application of metaheuristics to software-centric tasks
has brought forth many state-of-the-art approaches, and is jointly known
as \Gls{SBSE}.
Approaches fixating on the testing phase of the \gls{SDLC}
belong to the \gls{SBST} branch of \gls{SBSE}.

Search-based methods generally involve more sophisticated
mechanisms than fuzzing.
Irrespective of the problem, a notion of \textit{fitness}
allows the comparison of any two (valid) solutions
on the basis of at least one \textit{objective}.
In the realm of \gls{SBST}, solutions may consist
of test cases, and the corresponding fitness
of a solution may be determined on the structural
coverage a solution (test case) achieves, with the 
underlying objective of uncovering bugs.
Search algorithms often change solutions by means of \textit{mutation}.
Mutation operators introduce small, possibly stochastic perturbations
to alter the representation of a candidate, implicitly
equipping the search space with a \textit{topology}.
Over the years, researchers developed successful variations
of this high-level framework, which in many cases are able
to satisfactorily solve combinatorial
problems in reasonable time \citep{mcminn2011search}.
The remainder of this subsection coarsely examines
some of the most widespread metaheuristic search approaches
and grounds them through an \gls{SBST} lens.
\newacronym{HC}{HC} {Hill Climbing}
\newacronym{LS}{LS} {Local Search}
\newacronym{SA}{SA} {Simulated Annealing}
\newacronym{TS}{TS} {Tabu Search}

\subsubsection{Local Search Metaheuristics}

\Gls{LS} is among the most straight-forward
metaheuristic frameworks.
\Gls{HC} is a simple implementation of \gls{LS}, 
that greedily exploits the neighborhood structure 
of a solution.
It consists of an iterated procedure that repeatedly
attempts to improve a solution by applying mutations \cite{selman2006hill}.
If the mutated candidate exceeds the fitness of the previous
solution, the former replaces the latter, concluding the iteration.
This basic strategy has been an effective tool for 
solving combinatorial problems for over half a century \cite{lin1965computer}.
Both greedy and stochastic variations of \gls{LS} exist, providing 
compromises between the quality of the final solution
and the time to convergence.
From an \gls{SBST} perspective, \citet{arcuri2009theoretical} shows
that \gls{LS} asymptotically outperforms \gls{RT}, but falls short of 
more sophisticated global search techniques.
\citet{harman2007theoretical} further reinforce this hierarchy with
empirical evidence.

\citet{miller1976automatic} have demonstrated the
first application of \gls{HC} in automated test data generation in 1976.
Since then, researchers developed several variations of the 
\gls{LS} framework, such as \Gls{TS} \cite{glover1990tabu} and \Gls{SA}.
Introduced by \citet{kirkpatrick1983optimization} in 1983, \gls{SA}
finds its inspiration in statistical mechanics.
\gls{SA} behaves similarly to \gls{LS}, but
additionally incorporates an annealing mechanic that enables
the stochastic exploration of candidates that exhibit a lower
fitness than previous solutions.
A cooling schedule governs this mechanisms, which generally encourages
a broader exploration of the domain early in the search procedure.

\newacronym{GS}{GS} {Global Search}
\newacronym{EA}{EA} {Evolutionary Algorithms}
\newacronym{GA}{GA} {Genetic Algorithms}
\newacronym{GP}{GP} {Genetic Programming}

\subsubsection{Evolutionary Algorithms}

While \gls{LS} and its variants
provide useful optimization frameworks, \Gls{GS} techniques often
enable more effective and robust domain sampling mechanisms \cite{mcminn2004search}.
Of the many \gls{GS} search algorithms developed over the years, \Gls{EA} have gained
the most notoriety within the \gls{SBSE} community \cite{harman2011software}.

\gls{EA}s are widely applicable and robust, population-based algorithms
that practitioners have employed in a broad range of engineering tasks \cite{harman2011software}.
Though its conceptual origins trace back to Alan Turing's "\textit{Computing Machinery and Intelligence}" \cite{turing1950computing}, it was John Holland \cite{jh1975adaptation}
who in 1975 formalized and cemented the modern \gls{EA} paradigm.
A plethora of specialized \gls{EA}s have emerged from a vast spectrum of 
research domains, but arguably the most widespread variation is the
class of \Gls{GA}.
\Cref{alg:sga} provides the pseudocode for the simple \gls{GA}
proposed by Holland.
This constitutes the foundation of specialized automated test case generation algorithms
for several tasks, including unit testing 
\cite{tonella2004evolutionary, panichella2017automated},
data flow testing \cite{girgis2005automatic}, and REST API testing
\cite{arcuri2019restful}.
Inspired by the theory of natural evolution,
standard \gls{GA} formulations share several core ingredients:
\begin{itemize}
\item A \textit{representation} of \textit{individuals},
	often referred to as \textit{chromosomes}.
	Each individual represents a solution to the underlying
	problem.
	Together, the collection of individuals present at
	a given iteration forms a \textit{population}.

\item A \textit{fitness} function that determines
	the quality of individuals. 
	The fitness function is generally linked
	to one or more underlying \textit{objectives}.

\item A \textit{variation} mechanism that alters
	the individuals in a population to create \textit{offspring}.
	Variation often consists of a \textit{mutation} operator
	that changes an individual's chromosome, and a \textit{crossover}
	operator that combines individuals to create new offspring.
	
\item A \textit{selection} mechanism that pressures the is responsible
	for pressuring the population toward better solutions. Selection
	is generally driven by the fitness of individuals and strongly
	influences the rate of convergence of the algorithm.
\end{itemize}

\begin{algorithm}
	\SetKwData{Left}{left}
	\SetKwData{This}{this}
	\SetKwData{Up}{up}
	\SetKwFunction{init}{InitializeAndEvaluatePopulation}
	\SetKwFunction{terminate}{ShouldTermiante}
	\SetKwFunction{offspring}{CreateAndEvaluateOffspring}
	\SetKwFunction{select}{SelectIndividuals}
	\SetKwInOut{Input}{Input}
	\SetKwInOut{Output}{Output}
	\Input{Population size $n$}
	\Output{$N$ solutions}
	\BlankLine
	\DontPrintSemicolon
	%\emph{special treatment of the first line}\;
	
	$t \leftarrow 1$\;
	$P_1 \leftarrow $ \init{$n$}\;
	\While{$\neg$ \terminate{$P_t$}}{
		$O_t \leftarrow$ \offspring {$P_t, n$}\;
		$P_{t+1} \leftarrow$ \select{$P_{t}, O_{t}, n$}\;
		$t \leftarrow t + 1$\;
	}
	\Return $P_t$\;

	\caption{Simple Genetic Algorithm}
	\label{alg:sga}
\end{algorithm}

\newacronym{CFG}{CFG}{Context-Free Grammar}

\section{Context-Free Grammars}

Grammars are powerful tools used to study, specify,
and understand languages. 
A \Gls{CFG} leverages a recursive structure to
specify a language's syntactic properties, without accounting for
contextual semantics.
\gls{CFG}s have become central to many fuzzers thanks to their ability 
to effectively constrain generative processes to
input that is syntactically meaningful for the target \gls{SUT}.
For the remainder of this study, we use the terms terms \textit{grammar} 
and \gls{CFG} interchangeably, and we adhere to the definition put forward
by \citet{sipser1996introduction}.

Formally, a \gls{CFG} is a 4-tuple~$G=(V, \Sigma, R, S)$,
with (i)~$V$ a finite set of \textit{variables} or \textit{symbols},
(ii)~$\Sigma$ a set of \textit{terminals} or \textit{terminal symbols} 
such that $\Sigma \cap V = \emptyset$,
(iii)~$R \subseteq V \times \Sigma \cup R$
a set of \textit{rules} or \textit{productions} that define the ways in which a variable can be \textit{expanded}, and (iv)~$S \in V$ a \textit{start symbol} or \textit{start variable}.

