% \addcontentsline{toc}{part}{Part III}
\chapter{\label{cha:discussion}Discussion}

This section discusses the totality of the study and contextualizes
potential implications.
We begin by analyzing the limitations of our approach in \Cref{sec:limits},
before addressing potential threats to the validity of the research in \Cref{sec:threats}.
\Cref{sec:future} provides recommendations regarding potential 
avenues of future expansion of this study.

\section{\label{sec:limits}Limitations}

Throughout this study, we generated, compiled, and analyzed
roughly 200,000 Kotlin files, and uncovered five categories
of defects affecting three components of the Kotlin compiler.
The Kotlin compiler team verified all of these errors and is
working on targeted fixes for several of them.
In this process, several non-trivial
underlying biases and restraints that
\kf~carries came to light.
In this section, we analyze both the practical and theoretical
limitations of the fuzzer, critically assess them, and propose solutions.

\paragraph{Language Features}
The set of language features that the fuzzer currently supports is limited
to functions, four types of expressions, and statements.
This limitation, which we imposed to ensure the feasibility of the study,
drastically limits the domain of code that \kf~can generate.
In practice, this restriction means that any defects that are related to core language features,
such as classes, are outside the reach of the current research prototype.
Extending the set of language features is paramount for enhancing
the practicality of the tool and for effectively exploring the Kotlin syntax.
In its current state, extending novel language features consists of simply
implementing either one or two additional Java classes and making them accessible
in parent \gls{CFG} nodes.

\paragraph{Context}
Throughout the study, the root context used for sampling
consisted of a limited collection of standard library files mostly containing exceptions,
throwables, as well as numeric, logical, and string types.
Besides severely limiting the number of objects and classes
that \kf~ can generate, this factor also affects the
set of language features that our tool can leverage.
For instance, \texttt{for} statements in Kotlin can only be applied
to constructs that implement specific methods, also known as \textit{container} classes.
In the current context, no such type is available, further limiting the syntax \kf~can leverage.

\paragraph{Generative Constraints}
To ensure the validity of generated code, we embed
numerous constraints and biases in the sampling process.
The implementation of these constraints has proven effective,
as all sampled code examined in this study was valid (i.e., either \texttt{K1} or \texttt{K2}
compiled the code, unless a \texttt{OOM} error occurred).
However, these constraints may drastically reduce the domain of code that \kf~ can generate,
thus possibly also missing out on additional defects.
For instance, constraints such as \textit{the return type of a function must be samplable in the current context}
may render much of the context useless.
While intuitively respecting such a constraint is sensible
(if there is no way to sample a type in the current context, it is impossible to write a function
that returns such a type), it could be that breaking this constraint
might lead to false negative defects.
The genetic recombination proves this by breaking one such constraint, which
lead to the nested function defect discussed in \Cref{sec:buganalysis}.
Exploring how generative constraints affect the quality of the generated code and whether
they lead to additional defects requires careful and incremental changes, that we do not cover in this study.

\paragraph{Representation and Genetic Operators}
The hierarchical \textit{fragment, snippet, block} representation
is sufficient to meaningfully separate code
based on complexity and scope, but lacks the precision needed to
explore certain corner cases.
For instance, neither snippets, nor blocks are recursive,
which means variation operators in \gls{GA} algorithms
are only able to operate on the highest scope
of the file.
This limitation leaves out numerous cases in which
code can be exchanged on smaller levels, in turn exploring a new dimension of the language.
A representation capable of capturing this level of detail
requires a recursive implementation of blocks and fragments,
the foundation of which is already implemented in the current implementation of \kf.

\paragraph{\gls{GA} Convergence}
The application of \gls{GA}-based formulations to fuzzing
raises the problem of convergence.
Especially visible in the cases of \gls{SODGA} and \gls{STPGA},
the problem stems from the fact that relatively stable sets of solutions
may emerge quickly in those \gls{GA}s.
Since longer fuzzing campaigns become wasteful and redundant in those
scenarios, a mechanism that breaks this stability would enable
longer fuzzing runs.
Two options to alleviate this problem include restart mechanisms to effectively reset the search process,
and altering the targets of proximity-based algorithms.

\section{\label{sec:threats}Threats to Validity and Reproducibility}

Threats to \textit{construct validity} stem from the connection between 
the practical measures employed to quantify theoretical aspects of the study.
In this regard, we use standard \gls{DT} procedures to quantify
compiler defects, and link uncovered defects to static, measurable properties of
the generated code.
We also employ common metrics of effectiveness and efficiency that
are standard practice throughout empirical software engineering research.

Threats to \textit{internal validity} regard factors that could lead
to confounding the causes of observed phenomena.
We take several measures to address the large degree of randomness
inherent to fuzzing approaches.
First, we individually assess each uncovered defect
and analyze the components of the fuzzer involved in its generation.
Second, we ensure the fairness of
the comparisons by sharing parameters
between different configurations with
default values from literature, where available.
We also perform the comparison using a single implementation
of the tool, that relies on the same sampling process and genetic operators.
Lastly, to assess the effectiveness and efficiency of our algorithms,
we repeat independent runs 10 times, and report average values
that are subject to standard statistical analyses.

Threats to \textit{conclusion validity} affect the relation between the available data
and the credibility of the conclusions we derive based on it.
To this end, we base our analysis on over 200,000 generated files,
and vary the essential hyperparameters of our algorithms to several
sensible values.
We additionally perform statistical tests to compare algorithms representative
of their respective class.
In particular, we base our performance conclusions on
the application of the Wilcoxon signed-rank test, which
is a statistical procedure that does not make any assumptions
about the distribution of the underlying data.

Threats to \textit{reproducibility} concern factors that might cause the application
of the same research methods to result in significantly divergent or conflicting observations.
To mitigate this, we supply the entire code base that implements \kf~ \cite{kotfuzz-impl},
in addition to the entire set of generated files and adjacent preprocessed data \cite{kotfuzz-data}.
We also provide extensive documentation
to detail our tool, its configuration, and its applicability.
To ensure that no environmental factors interfere with the fuzzer,
we use containerization to isolate the dependency management and runtime of \kf.

\section{\label{sec:future}Future Work}

On top of addressing the limitations underlined in \Cref{sec:limits} and exploring
additional heuristics in the context of compiler fuzzing,
there are several directions future work could follow to extend this study.

\paragraph{Compiler Integration}
The heuristic fitness
criteria explored in this study work as methods
to guide the generative sampling process towards
promising areas of the Kotlin code space.
In parallel to those methods, one could leverage compiler information
directly into the search algorithm.
The compiler service described in \Cref{cha:tool} provides
an interface for this by allowing the fuzzer
to query for files that trigger faults according to \gls{DT}.
However, the fuzzer does not currently integrate this information into the search process.
Integration could follow different blueprints, such as maintaining code that uncovers faults as a permanent part of the population,
or avoiding generating novel blocks that resemble
known faulty code.
The promotes generating more instances of the same bug,
while the latter prevents it.

\paragraph{Defect Clustering at Runtime}
In its current implementation, \kf~only analyzes the generated files
after the fuzzing process has terminated.
We perform defect analysis automatically by parsing cached compiler messages
for common patterns, and manually analyze files that do not fit known patterns.
Heuristically performing this task at runtime could provide additional information to the fuzzer
in such a way that \kf~could orient the search process away from proeminent clusters,
thus better exploring the search space.
This process could take several forms.
For instance, the aforementioned compiler integration could be extended
to incorporate additional clustering capabilities.
Additionally, the embedding process could act as a mapping
tool that allows the application of standard clustering algorithms
without the need for compilation.
Both options would incur significant computational overhead in relation to \gls{RS}.

\paragraph{Integration with Mutation Fuzzing} \citet{stepanov2021type} introduced
a mutation-based fuzzer for Kotlin, that alters input code
in a sound and type-aware manner.
Since the variation operators currently available in \kf~are both vastly
different and less powerful than those in the mutation fuzzer,
exploring the integration of the two could give rise to new,
otherwise unattainable pieces of code.
This integration could follow a two-step approach,
where files generated through \kf~serve as input to the mutation fuzzer.
Alternatively, a closer integration of the two methods
could see \kf~use the mutation fuzzer as a powerful variation operator
implemented in one of its \gls{GA}s.

\paragraph{Specialized Embeddings}
The semantic proximity \gls{GA}s rely on embeddings to
guide the search process.
Throughout this study, we rely on \textsc{CodeBERT} \citet{feng2020codebert} as a
tool for providing this critical piece of the algorithm.
However, this choice might not be optimal.
First, newer models have superseded \textsc{CodeBERT} on several
\gls{ML4SE} tasks, and leveraging these newer solutions might increase
the quality of the embeddings.
Second, \textsc{CodeBERT} is not specifically pre-trained for Kotlin-related tasks.
Given the recent rise in popularity of Kotlin, there is likely
sufficient available open-source data to either pre-train
or fine-tune a large code model specifically for Kotlin.
In addition to increasing the quality of the embeddings,
such a model could prove useful for many other tasks
centered around Kotlin.
