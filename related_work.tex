\chapter{\label{cha:related}Related Work}

This chapter surveys relevant academic work within
the fields of compiler testing and fuzzing.
Section \ref{sec:compiler_testing} examines existing
automated compiler testing tools and their underlying rationale.
Section \ref{sec:specification_fuzzing} surveys algorithms
for specification fuzzing, a problem whose elemental challenges
share many commonalities with compiler testing. 

\section{\label{sec:compiler_testing}Automated Compiler Testing}

\newglossaryentry{tpg}
{
        name=test program generation,
        description={A compiler testing technique
        that generates novel valid programs
        without relying on a pre-existing corpus}
}

\newglossaryentry{pm}
{
        name=program mutation,
        description={A compiler testing technique
        that generates alters existing programs}
}

Over the years, specialized testing approaches have emerged
as tools for improving compiler reliability.
To navigate the ample space of approaches, we adopt a simplified
version of the taxonomy proposed by \citet{chen2020survey}. 
Most notably, we distinguish between \gls{tpg} and \gls{pm} approaches, discussed
in subsections \ref{subsec:tpg} and \ref{subsec:pm}, respectively.
The former generates programs for scratch, without any
external seed, while the latter mutates existing programs to generate
new variations. 
\Gls{tpg} approaches also differ in the way in which they
utilize the grammar of the target language.
Based on this property, \citet{chen2020survey} distinguish three
categories.
Grammar-directed approaches solely rely on the
language grammar to generate novel test cases.
Grammar-aided approaches heuristically exploit a given
grammar, which is often enriched with semantically
significant context. 
Fuzzers may not directly rely on a grammar,
and instead define sound heuristic that respect
a given linguistic syntax.
Finally, we distinguish a third, separate category of fuzzers
that fundamentally rely on machine learning techniques to generate input.
Since these fuzzers do not fit in with the standard taxonomy, we address them separately in


\newacronym{UB}{UB} {Undefined Behavior}
\newacronym{AST}{AST} {Abstract Syntax Tree}

\subsection{\label{subsec:tpg}Test Program Generation}

\citet{purdom1972sentence} devised one of the first algorithms aimed at
validating the validity of \gls{CFG} parsers.
This grammar-direct approach applies iterative rewriting rules
starting from a starting symbol, eventually reaching all grammar productions.
The algorithm heuristically favors short sentences, that aid in the
debugging of uncovered faults.
Faults are detected when the production sequence used to generate a particular
sentence differs from the representation returned by the parser under test.
Purdom's \cite{purdom1972sentence} empirical analysis demonstrates
that this algorithm is effective at exploring the available
states and transitions of \gls{CFG}s.
However, this simplistic approach does not generalize
well to contemporary programming language compiler testing,
as it contains no mechanism to account for the rich semantics
that accompany \gls{CFG} specifications.

\citet{yang2011finding} propose \textsc{Csmith}, a grammar-aided
test program generation tool aimed at finding bugs in commercial 
C compiler using \gls{DT}.
\textsc{Csmith} first creates a set of \texttt{struct} type
declaration, with a stochastically-driven  number of members, which 
it stores as additional information next to the C programming language grammar.
\textsc{Csmith} then creates programs in a top-down fashion, starting from a
single program entry point, the \texttt{main} function.
At each iteration, it interweaves two steps: (i) instantiating
a randomly chosen production from its grammar, accompanied
by target completion, and (ii) validating the intermediate program
using a series of safety checks with the goal of avoiding undefined behavior.
The generation process follows an "on-demand" paradigm, where \texttt{struct}
members and functions are generated when previously generated expressions
access them.
This approach fundamentally differs to Purdom's \cite{purdom1972sentence} paradigm,
in that it strongly emphasizes the semantic constraints of C over 
structural \gls{CFG} coverage.
\textsc{Csmith}'s effectiveness has inspired several variations such in
Lidbury et al.'s \textsc{CLsmith} \cite{lidbury2015many} and
Morisset et al.'s \cite{morisset2013compiler} tool for uncovering
concurrency bugs.

\citet{livinskii2020random} introduce \textsc{YARPGen}, a C/C++ program generation
tool that aims to increase the expressiveness and diversity of test programs.
\textsc{YARPGen} proceeds in top-down fashion and does not explicitly utilize
a formal language grammar.
Instead, it tracks a \textit{type environment} that stores all visible composite
data types, which implicitly guide the generative process.
The environment is iteratively enriched with newly generated types,
each having access to all previous entries.
Once established, the generated types drive the creation of global variables
that can be used in future statements.
\textsc{YARPGen} incrementally builds functions that utilize 
the previously generated types and symbols.
Each expression in a statement is recursively constructed 
such that it avoids \Gls{UB}.
To further increase the soundness of its output, \textsc{YARPGen}
performs local fixes that eliminate several common \gls{UB} instances.
Finally, \textit{generation policies} provide a mechanism
for adapting the probability distribution dictating
\textsc{YARPGen}'s choices.
This feature serves as a way of incorporating prior knowledge into
the program and enables users to target specific features of the language
or compiler.

\citet{holler2012fuzzing} developed \textsc{LangFuzz}, a grammar-directed fuzzing
approach that combines generative and mutative processes.
The mutation process follows a two-phase approach, first \textit{learning}
code fragments from existing test suites, and then constructing executable
mutants from its input. 
A grammar enables the extraction of code fragments, which \textsc{LangFuzz}
obtains by breaking down hand-written test suites.
In practice, a \textit{fragment pool} emerges from the non-terminal
instantiations of grammar rules within the input test suites.
To complement this mutative process, \textsc{LangFuzz} leverages
a generative procedure that stochastically expands the input
grammar in a breadth-first manner.
The two approaches work in tandem, comprising a trade-off
between the flexibility of the input grammar and the
rich semantics of test suites.

\citet{han2019codealchemist} introduce the notion of \textit{semantics-aware assembly},
which they implement in the \textsc{CodeAlchemist} tool, aimed at JavaScript
engine testing.
The key concept in semantics-aware assembly consists of building blocks,
also referred to as \textit{code bricks}, a notion similar to
\textsc{LangFuzz}'s \textit{code fragments} \cite{holler2012fuzzing}.
A code brick consists of a valid JavaScript \gls{AST} that is additionally
annotated with an \textit{assembly constraint}.
Assembly constraints are sets of pre- and post-condition
that must be satisfied for two code bricks to be compatible.
Such constraints specify what variables must be defined in
preceding code bricks, and what additional objects are
visible in subsequent bricks, respectively.
\textsc{CodeAlchemist} produces fuzzed programs by splitting
input seeds into code bricks, which it then stochastically
interconnects in ways that adhere to the bricks'
assembly constraints.

\newacronym{EMI}{EMI} {Equivalence Modulo Inputs}
\newacronym{MCMC}{MCMC} {Markov Chain Monte Carlo}

\subsection{\label{subsec:pm}Program Mutation}

\citet{le2014compiler} introduce the notion of \Gls{EMI},
a framework for establishing semantic equivalence between
programs.
Given a set of valid program inputs $I$,
two programs $P_1$ and $P_2$ are equivalent modulo $I$
if and only if their outputs are identical for all inputs
$i \in I$.
This notion lends itself naturally to the \gls{DT} framework:
since $P_1$ and $P_2$ are expected to produce identical results
for all input in $I$, any discrepancies that emerge during execution
can be attributed to compiler faults.
\textsc{Orion} is a mutation-centric realization of this framework for 
the C programming language.
To derive semantically equivalent programs, \textsc{Orion} profiles the
execution of a program $P$ over an input set $I$ and determines
which code snippets remain are not executed.
\textsc{Orion} then exploits this information to stochastically
prune the unexecuted lines of code, maintaining the semantics of
the original program $P$.

\citet{le2015finding} extend \textsc{Orion} in a tool named \textsc{Athena},
equipped with a heuristic guidance mechanism.
Still reliant on the \gls{EMI} framework, \textsc{Athena} improves
the random mutation-driven search of \textsc{Orion} with heuristic
diversity-based probability distribution.
The goal of this strategy is to exercise an ample range of compiler
optimization mechanics, which is more likely as when the equivalent 
programs themselves are more diverse.
To achieve this, \textsc{Athena} leverages a \Gls{MCMC} sampling technique.
A \Gls{CFG}-based similarity metric guides the \gls{MCMC} search,
favoring programs that significantly differ in their \gls{CFG} representation.
Novel programs emerge from either the deletion of unused code, or the insertion
of code in unused sections.

\newacronym{SPE}{SPE}{Skeletal Program Enumeration}

\citet{zhang2017skeletal} introduce the notion of \gls{SPE}, a framework
for exhaustively representing variable usage patterns in small programs.
\gls{SPE} breaks a program down into three components: a syntactic skeleton,
a collection of variable placeholders, and a set of variables.
Variable usages within a program are conceptually transformed into \textit{holes},
that can be \textit{filled} to \textit{realize} a concrete program.
This framework enables the enumeration of all possible arrangements
of variables in such a way that filling the holes results in novel programs.
The authors further define the concept of $\alpha$\textit{-equivalence} between
programs.
Two programs are $\alpha$-equivalent if they exhibit the same control- and data-
dependence properties.
Notably, naive enumeration can result in numerous $\alpha$-equivalent programs
under different variable permutations.
To avoid the generation of such equivalent programs due to $\alpha$\textit{-renaming},
the authors show the \gls{SPE} can be reduced to the \textit{set partition problem},
which displays the same equivalence structure.
The proposed algorithm handles variable scopes by promoting certain
\textit{local holes} to global ones and computing the Cartesian product 
between the two disjoint sets.

\newacronym{TCE}{TCE}{Type-Centric Enumeration}

\citet{stepanov2021type} propose \Gls{TCE}, an extension of \gls{SPE} 
that strongly emphasizes types as a powerful medium for generating
semantically diverse programs.
Similarly to \gls{SPE}, \gls{TCE} deconstructs a program into a 
skeleton, a collection of placeholders, and a set of variables.
In addition to this breakdown, however, \gls{TCE} equips each 
placeholder with an adherent type, and further considers a set
of \textit{callabels} that it uses alongside variables.
The algorithm proceeds in two phases.
The \textit{generation} phase is responsible for the 
creation of a set of typed expressions, which it extracts from
the callable set of its seed program.
The \textit{mutation} phase operates on a different seed program,
which it first merges with that of the previous stage.
Subsequently, the expressions obtained in the generation phase
fill type-compatible placeholders placed in the latter seed program.
Additional expressions obtained either from the standard library 
or primitive types are additionally employed to populate
remaining unfilled placeholders.

\subsection{\label{subsec:ml-fuzzers}Machine Learning-Based Fuzzers}

\newacronym{LSTM}{LSTM}{Long Short-Term Memory}
\newacronym{RNN}{RNN}{Recurrent Neural Network}
\newacronym{NAM}{NAM}{Neural Attribute Machines}

\citet{amodio2017neural} develop an approach called \gls{NAM},
that utilizes a \gls{LSTM} \cite{hochreiter1997long}
neural network to embed multifaceted constraints
and structural context into neural program generation.
Their method extends the \textit{attribute grammar}
formalism by leveraging an \gls{LSTM} that is capable
of generating \gls{AST} nodes of test programs.
The \gls{NAM} architecture exploits a tri-level
loss function that seeks to minimize 
(i) the model's divergence from the training set distribution,
(ii) the model's likelihood of violating semantic constraints, and
(iii) the model's likelihood of generating structurally valid, but nuanced incorrect code.
Training occurs on the level of \gls{AST} nodes,
additionally equipped with surrounding context and encoded
by the \gls{NAM}'s \textit{logical unit}.
Experiments show that structure emerges from the training procedure,
but the likelihood of breaking common language ordering and type-safety
constraints remains meaningful.

\citet{cummins2018compiler} build a test program-based generative fuzzer
that relies on a neural network for code generation.
The system depends on a corpus of code repositories
that serve as training data for the neural network.
At the core of the system resides a \gls{LSTM}
that models the distribution of vocabulary tokens within the input corpus.
To generate programs, the \gls{LSTM} is sampled to produce
novel vocabulary words in a token-by-token fashion,
which are first decoded before being packaged together into viable test programs.
The authors' empirical analysis shows that despite having no direct
model for the language structure, the tool, called \textsc{DeepSmith},
is able to generate highly structured code and uncover dozens of bug in the
\textsc{OpenCL} system.

We also note the existence of many statistics- and machine learning-
driven approaches for tangential tasks, such as in the fields of
program synthesis \cite{gulwani2017program} and
program completion \cite{raychev2014code}.
These tasks share similarities with the challenges of this
study, namely the generation of valid and contextually-aware code.
Though extendable to the field of compiler testing,
these methods additionally account for \textit{user intent}
(i.e., the goal of the program a user is attempting to synthesize).
Compiler fuzzers broadly disregard this aspect, as their goals differs from 
the former categories.

\section{\label{sec:specification_fuzzing}Specification Fuzzing}

\citet{steinhofel2022input} propose the Input Specification Language (\textsc{ISLa}),
a declarative language for integrating context-dependent semantic constraints
in otherwise coarser \gls{CFG} definitions.
Programming languages often impose complex semantic relations that 
\gls{CFG}s cannot encapsulate.
This includes even commonplace constraints such as \textit{"a variable must be
declared before it is assigned"}.
\textsc{ISLa} builds upon constraint solvers to incorporate complex annotations
that can model such relations. 
The \textsc{ISLa} solver iteratively expands a tree representation
each constraint annotation applied to the grammar according to production rules.
Expansions that do not match the annotations are excluded from the
cost function-guided solving procedure, as valid candidates emerge.
Eventually, if a the solver discovers a matching expansion, it first
(approximately) translates it to a regular expression representation
before querying a specialized constraint solver.
\textsc{ISLa} then converts this solution back to a 
derivation tree representation and substitutes it in the
original grammar graph.

\citet{havrikov2019systematically} study coverage criteria of \gls{CFG}s and 
accompanying coverage-centric generative algorithms.
\textit{Symbol coverage} is a straight-forward metric for evaluating
how much of a specification grammar a derivation tree covers.
Symbolic coverage is simply measured the fraction of symbolic nodes in the
grammar, that the derivation tree covers.
\textit{k-Path} coverage is a more nuanced criterion that additionally
takes into account the \textit{context} surrounding a symbolic node's
position within a tree. 
Formally, a \textit{k-path} is a directional sequence of nodes in a derivation 
that contains exactly \textit{k} symbolic nodes.
The concept of \textit{k-path} coverage naturally follows from this
definition: a collection of inputs achieves \textit{k-path} coverage if
each \textit{k-path} in the grammar graph is present in the underlying
set of derivation trees.
The authors derive an enumeration-based \textit{k-path} algorithm
stochastically iterates through all \textit{k-paths} up to a user-defined
depth and instantiating derivation trees.

\citet{kreutzer2020language} introduce the language specification language
(\textsc{LaLa}), a framework for semantically enriching the syntactic
specification of languages.
At its core, \textsc{LaLa} offers a declarative implementation of an
attribute grammar that operates over programs represented as \gls{AST}s.
Each production in the underlying syntactic \gls{CFG} takes the form of a
class, that embeds additional mechanisms to provide semantics to the
overarching \gls{AST}.
\gls{AST}s model information flow both bottom-up (through
\textit{synthesized} attribute values) and top-down (through
\textit{inherited} attribute values), and their semantic validity
is determined through boolean-valued \textit{guard} attributes embedded in
some nodes.
The \textsc{*Smith} fuzzer leverages this representation to generate valid
program representations through \gls{AST}s with conforming guard attributes.
The algorithm generates programs by randomly instantiating nodes by randomly
walking over available productions, additionally caching \textit{fail patterns}
that occur when nodes with false guard attribute values emerge.
Additional heuristics for fail pattern recognition, as well as subtree
replacement and deconstruction additionally improve the effectiveness
of \textsc{*Smith}.